\documentclass[12pt]{article}

\usepackage[utf8]{inputenc}
\usepackage[T1]{fontenc}
\usepackage[polish]{babel}
\usepackage{lmodern}
\usepackage{geometry}
\geometry{margin=1in}
\let\lll\undefined
\usepackage{amsmath,amssymb,amsthm}
\usepackage{enumitem}
\usepackage{xcolor}
\usepackage{hyperref}

% --- Środowiska ---
\newtheorem{defi}{Definicja}
\newtheorem{theorem}{Twierdzenie}
\newtheorem{remark}{Uwaga}
\newtheorem{prop}{Własność}
\newtheorem{ex}{Przykład}
\newtheorem{exer}{Zadanie}
\newcommand{\Sp}{\mathrm{Sp}}

\title{Rozkład Williamsona}
\author{}
\date{}

\begin{document}
\maketitle

\section*{Twierdzenie Williamsona}

\begin{theorem}[Williamsona]
Niech $M\in\mathbb{R}^{2n\times 2n}$ będzie symetryczna i dodatnio określona.
Istnieje macierz symplektyczna $S$ taka, że
\[
S^\top M S=\operatorname{diag}(\nu_1,\dots,\nu_n,\nu_1,\dots,\nu_n),\qquad \nu_i>0.
\]
Liczby $\nu_i$ nazywamy wartościami symplektycznymi $M$.
\end{theorem}

\begin{proof}[Szkic]
(1) Weź pierwiastek macierzowy $M^{1/2}$ (istnieje, bo $M\succ0$).\\
(2) Rozważ $A:=M^{1/2} J M^{1/2}$; $A$ jest antysymetryczna.\\
(3) Istnieje ortogonalna $O$ taka, że 
$O^\top A O=\bigoplus_{i=1}^n \begin{bmatrix}0&\nu_i\\ -\nu_i&0\end{bmatrix}$ 
z $\nu_i>0$.\\
(4) Z powyższego wynika, że istnieje $S$ symplektyczna spełniająca $S^\top M S=\operatorname{diag}(\nu_1,\dots,\nu_n,\nu_1,\dots,\nu_n)$.
\end{proof}

\begin{remark}[Uwaga do kroku z $M^{1/2}JM^{1/2}$]
(1) \emph{Antysymetryczność.} Dla $A:=M^{1/2}JM^{1/2}$ mamy
\[
A^\top=(M^{1/2}JM^{1/2})^\top=M^{1/2}J^\top M^{1/2}
= M^{1/2}(-J)M^{1/2}=-A,
\]
zatem $A$ jest macierzą antysymetryczną.

(2) \emph{Związek ze Schurem i blokami $2\times2$.} Każda rzeczywista macierz antysymetryczna
jest ortogonalnie podobna do sumy prostych bloków
$\begin{bmatrix}0&\nu_i\\ -\nu_i&0\end{bmatrix}$ z $\nu_i>0$, tzn. istnieje
ortogonalna $O$ taka, że
\[
O^\top A O=\bigoplus_{i=1}^n \begin{bmatrix}0&\nu_i\\ -\nu_i&0\end{bmatrix}.
\]
Jest to szczególny przypadek rzeczywistej postaci Schura (dla macierzy
antysymetrycznych bloki mają dokładnie powyższą formę).

(3) \emph{Jak odczytać $\nu_i$ bez jawnego Schura.} Ponieważ
\[
A=M^{1/2} J M^{1/2}=M^{1/2}(JM)M^{-1/2},
\]
macierze $A$ i $JM$ są podobne, więc mają te same wartości własne. Zatem
\[
\det(A-\lambda I)=\det\!\big(M^{1/2}(JM-\lambda I)M^{-1/2}\big)
=\det(JM-\lambda I).
\]
Stąd $\sigma(JM)=\{\pm i\nu_1,\dots,\pm i\nu_n\}$ i równoważnie
\[
\nu_i=\sqrt{\lambda_i\!\left(-(JM)^2\right)},
\]
czyli dodatnie pierwiastki wartości własnych $-(JM)^2$. To uzasadnia, że liczby $\nu_i$ ze
schodkowej postaci $A$ są wyznaczalne spektralnie bez przeprowadzania
jawnej redukcji blokowej.

(4) \emph{Domknięcie konstrukcji (intuicja).} Z punktu (2) mamy
$O^\top A O=\bigoplus_i J(\nu_i)$, gdzie $J(\nu):=\begin{bmatrix}0&\nu\\ -\nu&0\end{bmatrix}$.
Po dalszej lokalnej skali $R:=\bigoplus_i \operatorname{diag}(\sqrt{\nu_i},\,\nu_i^{-1/2})$
zachodzi
\[
R^\top\!\left(\bigoplus_i J(\nu_i)\right)R
= \bigoplus_i \nu_i\,J
= J\,\operatorname{diag}(\nu_1,\dots,\nu_n,\nu_1,\dots,\nu_n).
\]
Stąd, dla $S:=M^{1/2}OR$ otrzymujemy $S^\top J S = J\,\mathrm{diag}(\nu_i,\nu_i)$,
co po dodatnim skalowaniu bloków prowadzi do postaci
Williamsona $S^\top M S=\operatorname{diag}(\nu_1,\dots,\nu_n,\nu_1,\dots,\nu_n)$.
\end{remark}

\section*{Zadania}

\begin{exer}
Wygeneruj macierz dodatnio określoną, a następnie oblicz wartości symplektyczne
dwoma metodami:
\begin{enumerate}[label=(\alph*)]
\item bezpośrednio ze spektrum $JM$,
\item z bloków $2\times 2$ w rozkładzie Schura macierzy $A=M^{1/2}JM^{1/2}$.
\end{enumerate}
Porównaj wyniki.
\end{exer}

\begin{verbatim}
symplectic_eigs <- function(M, tol = 1e-12) {
  # J = [[0, I],[ -I, 0 ]]
  n <- nrow(M) / 2
  J <- rbind(cbind(matrix(0, n, n), diag(n)),
             cbind(-diag(n), matrix(0, n, n)))
  ev <- eigen(J %*% M, only.values = TRUE)$values
  # eigenvalues come in ± i*nu; take positive nus
  nus <- sort(round(Mod(Im(ev[Im(ev) > 0])), 12))
  nus[ nus > tol ]
}
\end{verbatim}

\begin{verbatim}
library(expm)

symplectic_eigs_schur <- function(M, tol = 1e-12) {
  n <- nrow(M) / 2
  J <- rbind(cbind(matrix(0, n, n), diag(n)),
             cbind(-diag(n), matrix(0, n, n)))
  # pierwiastek macierzowy przez SVD
  sv <- svd(M)
  Msq <- sv$u %*% (diag(sqrt(sv$d))) %*% t(sv$u)
  A <- Msq %*% J %*% Msq
  S <- Schur(A)
  T <- S$T
  T[Mod(T) < tol] <- 0
  i <- 1; nus <- c()
  while (i <= nrow(T)) {
    if (i < nrow(T) && T[i+1, i] != 0) {
      b <- T[i, i+1]; c <- T[i+1, i]
      nus <- c(nus, sqrt(-b * c))
      i <- i + 2
    } else {
      i <- i + 1
    }
  }
  sort(Re(nus))
}
\end{verbatim}

\section*{Przykład liczbowy (n=2)}

Weźmy macierz dodatnio określoną w postaci blokowo–diagonalnej
\[
M=\begin{bmatrix}
2 & 1 & 0 & 0\\
1 & 3 & 0 & 0\\
0 & 0 & 2 & 1\\
0 & 0 & 1 & 3
\end{bmatrix}.
\]
Jej wartości symplektyczne wynoszą numerycznie
\[
\nu_1 \approx 1.381966011,\qquad
\nu_2 \approx 3.618033989.
\]

\noindent Poniższy kod w~R potwierdza wynik dwiema metodami (spektrum $JM$ oraz
bloki Schura dla $A=M^{1/2}JM^{1/2}$):

\begin{verbatim}
## M z przykładu
M <- matrix(c(2,1,0,0,
              1,3,0,0,
              0,0,2,1,
              0,0,1,3), 4, 4, byrow = TRUE)

## (a) z definicji: ze spektrum J M
symplectic_eigs(M)
# oczekiwane: 1.381966011, 3.618033989

## (b) przez bloki 2x2 w rozkładzie Schura dla A=M^{1/2} J M^{1/2}
# (wymaga pakietu expm)
symplectic_eigs_schur(M)
# te same wartości (w granicach błędu numerycznego)
\end{verbatim}

\noindent Dla tej klasy przykładów (identyczne bloki $2\times2$ na diagonali)
obydwie metody zwracają te same $\nu_i$ z~dokładnością numeryczną.

\begin{remark}[Mini-uwaga kontrolna]
Dla macierzy dodatnio określonych $M$ typu kowariancyjnego zachodzi
\[
\prod_{i=1}^n \nu_i = \sqrt{\det M}.
\]
W naszym przykładzie
\[
\sqrt{\det M} = \sqrt{25} = 5,
\qquad
\nu_1\nu_2 \approx 1.381966011 \cdot 3.618033989 = 5,
\]
co potwierdza poprawność obliczeń.
\end{remark}

\begin{quote}
\small
\emph{The message of Williamson’s theorem is that one can diagonalize any
positive definite symmetric matrix $M$ using a symplectic matrix, and that the
diagonal matrix has the very simple form
\[
D=\begin{bmatrix} \Xi & 0 \\ 0 & \Xi \end{bmatrix},
\]
where the diagonal elements of $\Xi$ are the moduli of the eigenvalues of $JM$.
This is a truly remarkable result which will allow us to construct a precise phase
space quantum mechanics in the ensuing Chapters. One can without exaggeration
say that this theorem carries in germ the recent developments of symplectic
topology; it leads immediately to a proof of Gromov’s famous non-squeezing
theorem in the linear case and has many applications both in mathematics and
physics. Williamson proved this result in 1963 and it has been rediscovered
several times since then with different proofs.}
\end{quote}

\begin{flushright}
Maurice de Gosson
\end{flushright}


\end{document}
