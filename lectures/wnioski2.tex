\documentclass[12pt,a4paper]{article}
\usepackage[utf8]{inputenc}
\usepackage[T1]{fontenc}
\usepackage{lmodern}
\usepackage{amsmath,amssymb}
\usepackage{geometry}
\geometry{margin=2.5cm}

\begin{document}

\begin{center}
\Large \textbf{Podsumowanie lekcji 2}
\end{center}

\bigskip

Dzisiejsze zajęcia pokazały, że:
\begin{itemize}
    \item Sprzeczne lub niedookreślone układy równań wymagają nowej definicji „rozwiązania”.
    \item Pseudoodwrotność Moore’a--Penrose’a dostarcza \textbf{kanonicznego wyboru}: rozwiązania najmniejszych kwadratów o minimalnej normie.
    \item Geometrycznie $x^\star=A^+b$ to rzut punktu $b$ na przestrzeń kolumn macierzy $A$.
    \item Pseudoodwrotność jest naturalnym narzędziem, gdy klasyczna odwrotność nie istnieje.
\end{itemize}

\bigskip

\textbf{Ocena rozmów studentów z AI.}  
Najważniejsze kryterium: \emph{jak głęboko potrafili pytać}.  
- Dociekliwe pytania o interpretację geometryczną, stabilność czy inne normy będą wysoko oceniane.  
- Samo „policz wynik” nie wystarczy.  

\bigskip

\begin{center}
\textit{„Matematyka zaczyna się tam, gdzie trzeba na nowo zdefiniować, co znaczy ‘rozwiązanie’.”}
\end{center}

\end{document}
