\documentclass[12pt,a4paper]{article}
\usepackage[utf8]{inputenc}
\usepackage[T1]{fontenc}
\usepackage{lmodern}
\usepackage{amsmath,amssymb}
\usepackage{geometry}
\geometry{margin=2.5cm}

\begin{document}

\begin{center}
\Large \textbf{Lekcja 2: Pseudoodwrotność i nowe prowokacje}
\end{center}

\section*{1. Wprowadzenie}
Na poprzedniej lekcji poznaliśmy sprzeczny układ równań i zauważyliśmy, że 
z jego rozwiązań cząstkowych powstaje trójkąt (convex polytope). 
Rozważaliśmy różne sposoby definiowania „środka” tego zbioru.

Dziś przechodzimy do \textbf{pseudoodwrotności} (Moore--Penrose inverse).

\section*{2. Pseudoodwrotność}
Dla macierzy $A\in \mathbb{R}^{m\times n}$ pseudoodwrotność $A^+$ 
jest zdefiniowana jako macierz $n\times m$ spełniająca cztery warunki Penrosea:
\[
AA^+A=A,\qquad A^+AA^+=A^+,\qquad (AA^+)^\top=AA^+,\qquad (A^+A)^\top=A^+A.
\]

\textbf{Własności:}
\begin{itemize}
    \item Jeśli $A$ jest odwracalna, to $A^+=A^{-1}$.
    \item Jeśli $A$ jest prostokątna, to $x=A^+b$ jest rozwiązaniem \emph{najmniejszych kwadratów} o minimalnej normie.
    \item Obliczeniowo: $A^+$ wyznaczamy z rozkładu SVD.
\end{itemize}

\subsection*{2a. Definicja metody najmniejszych kwadratów}
Dany jest układ równań
\[
A x \approx b, \quad A\in \mathbb{R}^{m\times n},\; b\in\mathbb{R}^m.
\]
Układ może być sprzeczny, czyli brak dokładnego $x$ spełniającego $Ax=b$.

\medskip
\noindent
\textbf{Definicja.} Rozwiązaniem \emph{najmniejszych kwadratów} nazywamy wektor
\[
x^\star = \arg\min_{x\in\mathbb{R}^n} \|Ax-b\|_2^2.
\]

\medskip
\noindent
\textbf{Warunek normalny.} Punkt $x^\star$ spełnia układ równań
\[
A^\top A x^\star = A^\top b.
\]

\medskip
\noindent
\textbf{Interpretacja geometryczna.} Rozwiązanie najmniejszych kwadratów $x^\star$ jest takie, że $Ax^\star$ to \emph{rzut ortogonalny} wektora $b$ na przestrzeń kolumn macierzy $A$.

---

\section*{3. Powrót do przykładu}
Dla macierzy
\[
A=\begin{bmatrix}2 & 1\\ 1 & -1\\ 1 & 1\end{bmatrix},\quad
b=\begin{bmatrix}5\\1\\2\end{bmatrix}
\]
otrzymujemy
\[
x^\star=(A^\top A)^{-1}A^\top b=\left(\tfrac{27}{14},\,\tfrac{5}{7}\right).
\]
Punkt ten leży wewnątrz trójkąta i można go interpretować jako „ważony środek” rozwiązań cząstkowych.

\section*{4. Zadania do pracy z AI}
Każdy student prowadzi rozmowę z ChatGPT i zapisuje ją w pliku tekstowym.

\subsection*{Zadanie 1. Układ niedookreślony}
Rozważ układ
\[
x+y=2,\quad 2x+2y=4.
\]
\begin{itemize}
    \item Dlaczego układ ma nieskończenie wiele rozwiązań?
    \item Jak pseudoodwrotność wybiera jedno z nich?
    \item Jaki sens ma rozwiązanie minimalnej normy?
\end{itemize}



\subsection*{Zadanie 2. Projekcje}
Dla macierzy
\[
A=\begin{bmatrix}2 & 1\\ 1 & -1\\ 1 & 1\end{bmatrix},\quad
b=\begin{bmatrix}5\\1\\2\end{bmatrix}
\]
Pokaż, że $AA^+$ jest macierzą projekcji ortogonalnej na obraz $A$.
\begin{itemize}
    \item Jaką interpretację ma $x^\star=A^+b$?
    \item Jak wygląda rzut punktu $b$ na przestrzeń kolumn $A$?
\end{itemize}

\subsection*{Zadanie 3. Zmiana normy}
Dla macierzy
\[
A=\begin{bmatrix}2 & 1\\ 1 & -1\\ 1 & 1\end{bmatrix},\quad
b=\begin{bmatrix}5\\1\\2\end{bmatrix}
\]
Zamiast minimalizować $\|Ax-b\|_2$, rozważ
\[
\min_x \|Ax-b\|_1.
\]
\begin{itemize}
    \item Jak zmienia się rozwiązanie?
    \item Czy pseudoodwrotność działa?
    \item Jakie metody są potrzebne?
\end{itemize}

\end{document}
