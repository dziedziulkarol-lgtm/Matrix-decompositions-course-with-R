\documentclass[12pt,a4paper]{article}
\usepackage[utf8]{inputenc}
\usepackage[T1]{fontenc}
\usepackage{lmodern}
\usepackage{amsmath,amssymb}

\begin{document}

\begin{center}
\Large \textbf{Instrukcja do ćwiczenia}
\end{center}

\bigskip

\noindent
W ramach dzisiejszych zajęć pracujemy z układem równań liniowych:
\[
\begin{cases}
2x + y = 5, \\
x - y = 1, \\
x + y = 2.
\end{cases}
\]

\bigskip

\textbf{Zasady pracy:}
\begin{enumerate}
    \item Każdy z Was prowadzi własną rozmowę z ChatGPT (AI).
    \item Rozmowę zapisujecie w pliku tekstowym.
    \item Oceniana jest przede wszystkim \emph{jakość i głębokość pytań}, nie sam „wynik”.
\end{enumerate}

\bigskip

\textbf{Na co zwrócić uwagę w rozmowie:}
\begin{itemize}
    \item Czy układ ma rozwiązanie? Co znaczy, że układ jest \emph{sprzeczny}?
    \item Jak zdefiniować „rozwiązanie” w takiej sytuacji?
    \item Co to znaczy „najlepsze przybliżenie” rozwiązania i jak je rozumieć geometrycznie?
    \item Jakie metody algebry liniowej mogą tu pomóc (np. formuła normalna, rozkłady)?
\end{itemize}

\bigskip

\textbf{Kryteria oceny rozmowy:}
\begin{enumerate}
    \item Precyzja i dociekliwość pytań.
    \item Próby zrozumienia \emph{dlaczego}, a nie tylko \emph{jak obliczyć}.
    \item Łączenie problemu z szerszym kontekstem (geometria, metody przybliżone, stabilność).
\end{enumerate}

\bigskip

\textit{Cel: pokazać, jak potraficie eksplorować problem matematyczny poprzez trafne pytania.}

\end{document}
