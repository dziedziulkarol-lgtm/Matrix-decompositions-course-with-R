\documentclass[12pt,a4paper]{article}
\usepackage[utf8]{inputenc}
\usepackage[T1]{fontenc}
\usepackage{lmodern}
\usepackage{amsmath,amssymb}

\begin{document}

\begin{center}
\Large \textbf{Instructions for the Exercise}
\end{center}

\bigskip

\noindent
Today we work with the system of linear equations:
\[
\begin{cases}
2x + y = 5, \\
x - y = 1, \\
x + y = 2.
\end{cases}
\]

\bigskip

\textbf{Rules of work:}
\begin{enumerate}
    \item Each of you conducts your own conversation with ChatGPT (AI).
    \item Save the conversation in a text file.
    \item What is graded is primarily the \emph{quality and depth of your questions}, not the final “answer”.
\end{enumerate}

\bigskip

\textbf{What to pay attention to during the dialogue:}
\begin{itemize}
    \item Does the system have a solution? What does it mean for a system to be \emph{inconsistent}?
    \item How can we define a “solution” in such a case?
    \item What does “best approximation” mean, both geometrically and computationally?
    \item Which tools of linear algebra can help here (e.g., normal equations, matrix decompositions)?
\end{itemize}

\bigskip

\textbf{Criteria for evaluation:}
\begin{enumerate}
    \item Precision and inquisitiveness of your questions.
    \item Attempts to understand \emph{why}, not only \emph{how to compute}.
    \item Ability to connect the problem with a broader context (geometry, approximation methods, stability).
\end{enumerate}

\bigskip

\textit{Goal: to demonstrate how you can explore a mathematical problem by asking meaningful questions.}

\newpage
\begin{center}
\Large \textbf{Class Summary}
\end{center}

\bigskip

We started from a seemingly simple task: “find the solution of a system of equations.”  
Adding a third equation made the system inconsistent and left us with no classical solution.

This opens important questions:
\begin{itemize}
    \item What does it mean to “solve” a problem when exact satisfaction of all equations is impossible?
    \item How should we understand the idea of the “best approximation” --- geometrically and computationally?
    \item Which tools of linear algebra allow us to formulate and find such solutions?
\end{itemize}

Today’s conversations with AI were meant to show that the key skill is the \emph{ability to ask questions}.  
Inquisitive questions lead to new concepts and methods, and thus to deeper understanding.

\bigskip
\begin{center}
\textit{“Mathematics begins where simple methods stop working.”}
\end{center}

\end{document}
