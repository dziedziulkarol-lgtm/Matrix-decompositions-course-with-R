\documentclass[12pt]{article}
\usepackage[utf8]{inputenc}
\usepackage[T1]{fontenc}
\usepackage{lmodern}
\usepackage{geometry}
\geometry{margin=1in}
\usepackage{amsmath,amssymb}
\usepackage{enumitem}
\usepackage{xcolor}
\usepackage{hyperref}

\title{Lekcja 3-4: Praktyczna powtórka – algebra liniowa}
\author{}
\date{}

\begin{document}
\maketitle

\begin{quote}
\emph{“Learn how MIT students learn — curiosity first, theory follows.”}
\end{quote}

\section*{Zadanie domowe (flipped classroom)}
Obejrzyj w wolnym czasie: \textbf{Wykład 1 MIT-a 18.06}, Gilbert Strang – „The geometry of linear equations” (MIT OCW).\\[0.5em]
Link: \href{https://ocw.mit.edu/courses/18-06-linear-algebra-spring-2010/resources/lecture-1-the-geometry-of-linear-equations/}{OCW 18.06, Lecture 1}.  

To doskonały wizualny wstęp do trzech interpretacji układów równań liniowych:
\begin{itemize}
  \item \emph{row picture} – linie w układzie równań,
  \item \emph{column picture} – kombinacje liniowe kolumn,
  \item \emph{matrix picture} – forma macierzowa \(Ax=b\).
\end{itemize}

\vspace{1em}


\vspace{1em}
\section*{Ściąga – Normy macierzowe}

 Niech $x=[x_1,\ldots,x_n]^T$, $y=[y_1,\ldots,y_n]^T$ wektory w $R^n$.
 Niech $A$ macierz rzeczywista wymiaru $n\times m$
 \[
 A=[a_{i,j}]_{1\leq i\leq n,1\leq j\leq m}
 \]
\begin{center}
\begin{tabular}{|c|c|}
\hline
Pojęcie & Wzór \\
\hline
Norma euklidesowa & $\|x\|_2 = \sqrt{\sum_i x_i^2}$ \\
Norma Frobeniusa & $\|A\|_F = \sqrt{\sum_{i,j} a_{ij}^2}$ \\
Norma operatorowa & $\|A\|_2 = \max_{\|x\|_2=1} \|Ax\|_2$ \\
Iloczyn skalarny & $\langle x, y \rangle = \sum_i x_i y_i$\\
\hline
\end{tabular}
\end{center}
\smallskip
\noindent
\emph{Uwaga:} Normę operatorową definiuje się zawsze względem ustalonej normy na wektorach.
Najczęściej przyjmuje się normę euklidesową, co prowadzi do powyższego wzoru.

\vspace{1em}
\section*{Instrukcja pracy z AI}
Każdy student wybiera jedno pojęcie i przygotowuje:
\begin{itemize}
  \item krótki opis (intuicja, geometria, zastosowanie),
  \item prosty przykład liczbowy,
  \item zapis w formacie \LaTeX.
\end{itemize}

\textbf{Cel:} tworzymy wspólną notatkę grupy, sprawdzamy jakość wyjaśnień AI i dyskutujemy nad nimi.

\vspace{1em}
\section*{Pojęcia do opracowania}

\subsection*{Dla wszystkich}
\begin{itemize}[label=$\triangleright$]
\item Rząd macierzy
\item Twierdzenie o jednoznaczności rozwiązań
\item Macierz odwrotna (gdy $\det(A)\neq 0$)
\item Rzut (projekcja) wektora na podprzestrzeń
\item Wyznacznik macierzy i jego interpretacja geometryczna
\item Macierz jako operator
$A:R^n \to R^m$, obraz operatora i jego jądro
\item Macierz zespolona jako operator
$A:C^n \to C^m$, obraz operatora i jego jądro
\item Macierz ortogonalna \(Q\): \(Q^\top Q = I\). Zachowuje długość i kąty (izometria).
\item Vandermonde matrix i interpolacja wielomianami pokazać związek
\item Macierze ortogonalne, przykład: macierz Hadamara
\end{itemize}
\subsection*{Dla chętnych (harder topics)}
\begin{itemize}[label=$\diamond$]

\item Macierz pseudoodwrotna i rozwiązywanie układów równań
\item Wektory własne i wartości własne 
\item Normy: Frobeniusa, operatorowa (dlaczego są ważne?)
\item Graf i jego macierz G. Strang page 129
\item Macierze stochastyczne
\item Non-negative matrix factorization (NMF) i przykłady aproksymacji
Nonnegative matrix approximation (NNMA)

\item Roundoff error G. Strang page 69-70
\[
A=\begin{bmatrix}1 & 1\\ 1 & 1.0001\end{bmatrix}.
\]
     Jak zmienia się rozwiązanie układu przy niewielkich perturbacjach?
    

\end{itemize}


\vspace{1em}
\section*{Lekcja 4: mini-prezentacje}
Każdy student przygotowuje dla grupy krótką prezentację (3–5 minut).  
Wyniki przesyłamy na stronę grupy przed zajęciami.

\vspace{1em}
\section*{Po co to wszystko?}
\begin{itemize}
  \item Żeby wyrównać szanse studentów z różnym zapleczem,
  \item Nie zaburzać tempa kursu formalną teorią,
  \item Promować model \emph{„ucz się, jak MIT”} — prestiż i motywacja.
\end{itemize}

\end{document}
