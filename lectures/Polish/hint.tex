\documentclass[12pt,a4paper]{article}
\usepackage[utf8]{inputenc}
\usepackage[T1]{fontenc}
\usepackage{lmodern}
\usepackage{amsmath,amssymb}
\usepackage{geometry}
\geometry{margin=2.5cm}

\begin{document}

\begin{center}
\Large \textbf{Lekcja 2 – Wskazówki obliczeniowe do Zadań 1–4}
\end{center}

\section*{Zadanie 1. Układ niedookreślony}
Układ:
\[
x+y=2,\qquad 2x+2y=4.
\]
\textbf{Wskazówki.}
\begin{enumerate}
  \item Druga równość jest wielokrotnością pierwszej $\Rightarrow$ zbiór rozwiązań:
  \[
  \{(x,y)\in\mathbb{R}^2:\; x+y=2\}=\{(t,\,2-t):\; t\in\mathbb{R}\}.
  \]
  \item Rozwiązanie minimalnej normy (w sensie Euklidesowym) znajdziesz z
  \[
  \min_{t\in\mathbb{R}} \; \|(t,\,2-t)\|_2^2 \;=\; t^2+(2-t)^2,
  \]
  co daje $t=1$ i punkt $(1,1)$.
  \item Dla pseudoodwrotności: jeśli $A=\bigl[\begin{smallmatrix}1&1\\2&2\end{smallmatrix}\bigr]$, $b=\bigl[\begin{smallmatrix}2\\4\end{smallmatrix}\bigr]$, to $x^\star=A^+b$ pokrywa się z rozwiązaniem minimalnej normy, tj. $x^\star=(1,1)$.
  \item Geometria: rozwiązania tworzą prostą $x+y=2$. Punkt minimalnej normy to rzut $(0,0)$ na tę prostą.
\end{enumerate}

\section*{Zadanie 2. Stabilność obliczeń}
Macierz:
\[
A_\varepsilon=\begin{bmatrix}1 & 1\\ 1 & 1+\varepsilon\end{bmatrix},\qquad \varepsilon=10^{-4}.
\]
\textbf{Wskazówki.}
\begin{enumerate}
  \item Kondycję w normie spektralnej można oszacować przez wartości osobliwe: 
  \(
  \kappa_2(A)=\dfrac{\sigma_{\max}(A)}{\sigma_{\min}(A)}.
  \)
  \item Dla $2\times 2$ wygodnie policzyć wartości własne macierzy symetrycznej $A^\top A$:
  \[
  A_\varepsilon^\top A_\varepsilon=
  \begin{bmatrix}
  2 & 2+\varepsilon\\
  2+\varepsilon & 2+2\varepsilon+\varepsilon^2
  \end{bmatrix}.
  \]
  \item Wartości własne $\lambda_{1,2}$ wyznacza wzór
  \(
  \lambda_{1,2}=\tfrac{\operatorname{tr}\pm\sqrt{(\operatorname{tr})^2-4\det}}{2}.
  \)
  Następnie $\sigma_{1,2}=\sqrt{\lambda_{1,2}}$ i $\kappa_2=\sqrt{\lambda_1/\lambda_2}$.
  \item Intuicja: gdy $\varepsilon\to 0$, macierz staje się bliska osobliwej, więc $\sigma_{\min}\to 0$ i $\kappa_2\to\infty$ — duża wrażliwość rozwiązań na perturbacje danych.
\end{enumerate}

\section*{Zadanie 3. Projekcje}
Teza: $P:=AA^+$ jest rzutem ortogonalnym na $\mathcal{R}(A)$.
\textbf{Wskazówki.}
\begin{enumerate}
  \item Z równań Penrosea: $(AA^+)^\top=AA^+$ (symetria) oraz
  \(
  (AA^+)^2=AA^+AA^+=AA^+ \text{ (po wstawieniu $A^+AA^+=A^+$)}.
  \)
  To znaczy: $P$ jest idempotentny i symetryczny $\Rightarrow$ rzut ortogonalny.
  \item Obraz: $\mathcal{R}(P)=\mathcal{R}(AA^+)=\mathcal{R}(A)$.  
  Jądro: $\mathcal{N}(P)=\mathcal{N}(A^\top)$.
  \item Interpretacja LS: $x^\star=A^+b$, a wektor dopasowany to $Ax^\star=AA^+b=Pb$, czyli rzut $b$ na przestrzeń kolumn $A$.
\end{enumerate}

\section*{Zadanie 4. Zmiana normy na $\ell_1$}
Problem:
\[
\min_x \|Ax-b\|_1.
\]
\textbf{Wskazówki.}
\begin{enumerate}
  \item Wprowadź zmienne pomocnicze $u,v\in\mathbb{R}^m_{\ge 0}$ z $Ax-b=u-v$ i minimalizuj $\mathbf{1}^\top(u+v)$:
  \[
  \begin{aligned}
  \min_{x,u,v}\quad & \mathbf{1}^\top u+\mathbf{1}^\top v\\
  \text{s.t.}\quad & Ax-b=u-v,\\
  & u\ge 0,\; v\ge 0.
  \end{aligned}
  \]
  To jest program liniowy.
  \item Pseudoodwrotność \emph{nie} rozwiązuje problemu $\ell_1$.  
  W praktyce stosuje się metody PL (simplex/interior-point) albo algorytmy typu iteracyjnej ważonej LS (IRLS).
  \item Geometria: kula $\ell_1$ to romb (poliedr), więc rozwiązania mają tendencję do „rzadkości” (sprzyja zerowym składowym residuów/parametrów).
\end{enumerate}

\end{document}
