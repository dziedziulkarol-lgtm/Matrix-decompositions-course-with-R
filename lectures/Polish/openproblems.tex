% !TEX program = pdflatex
\documentclass[11pt,a4paper]{article}
\usepackage[utf8]{inputenc}
\usepackage[T1]{fontenc}
%\let\lll\undefined
%\usepackage[polish]{babel}
%\usepackage{polski}
\usepackage{lmodern}
\usepackage{amsmath,amssymb,amsthm,mathtools}
\usepackage{geometry}
\usepackage{microtype}
\usepackage{enumitem}
\usepackage{xcolor}
\usepackage{hyperref}
\hypersetup{colorlinks=true, linkcolor=blue!50!black, urlcolor=blue!50!black, citecolor=blue!50!black}
\geometry{margin=2.6cm}

% Theorem-like environments
\theoremstyle{definition}
\newtheorem{openproblem}{Problem otwarty}
\newtheorem*{remark}{Uwaga}

% Handy macros
\DeclareMathOperator{\rank}{rank}
\DeclareMathOperator{\srank}{srank}
\DeclareMathOperator{\diag}{diag}
\DeclareMathOperator{\tr}{tr}
\DeclareMathOperator{\spec}{spec}
\newcommand{\RR}{\mathbb{R}}
\newcommand{\CC}{\mathbb{C}}
\newcommand{\NN}{\mathbb{N}}

\title{Lekcja 5. Otwarte problemy związane z macierzami\\\large szkic do dyskusji ze studentami (stan na sierpień 2025)}
\author{Zebrane i skomentowane przez prowadzących}
\date{}

\let\lll\undefined
\usepackage[polish]{babel}
\begin{document}
\maketitle

\begin{abstract}
Celem tej notatki jest zebranie kilku klasycznych i współczesnych problemów otwartych, w których macierze odgrywają rolę główną. Każdy wpis zawiera \emph{krótkie sformułowanie}, \emph{dlaczego to ważne}, \emph{przykład}, oraz \emph{wskazówki do lektury startowej}. Na końcu dodajemy propozycje mini–projektów dla studentów.\footnote{Status niektórych problemów szybko ewoluuje; przed głębszym użyciem zalecamy sprawdzenie aktualnego stanu literatury.}
\end{abstract}

\tableofcontents
\bigskip

\section{Spektra i problemy odwrotne}

\begin{openproblem}[Nonnegative Inverse Eigenvalue Problem (NIEP)]\label{prob:NIEP}
\textbf{Sformułowanie.} Scharakteryzować zbiory liczb zespolonych, które są widmem pewnej macierzy \emph{nieujemnej}. W szczególności pełna charakterystyka dla przypadków rzeczywistych/symetrycznych pozostaje otwarta w ogólności.

\textbf{Przykład.} Macierz
\[
A = \begin{bmatrix} 0.5 & 0.5 \\ 0.5 & 0.5 \end{bmatrix}
\]
ma wartości własne \(1,0\) i jest nieujemna. Pytanie: dla jakich zestawów liczb można zbudować podobne macierze?

\textbf{Dlaczego to ważne?} Połączenia z teorią Perrona–Frobeniusa, modelowaniem sieci i dynamiką populacji; prowadzi do subtelnych ograniczeń kombinatorycznych.

\textbf{Lektura startowa.} Klasyczne przeglądy NIEP oraz nowsze wyniki dla małych wymiarów i klas specjalnych (symetryczne, dwustosunkowe, itp.).
\end{openproblem}

\section{Rozkłady niskiego rzędu i selekcja kolumn (SVD, CUR, RRQR)}

\begin{openproblem}[Dobór kolumn/wierszy o gwarancjach względnego błędu]\label{prob:CSSP}
\textbf{Sformułowanie.} W \emph{CSSP/CUR} znaleźć minimalną liczbę kolumn/wierszy (w funkcji \(k,\varepsilon\)), które deterministycznie dają aproksymację rzędu \(k\) o błędzie \((1+\varepsilon)\) względem SVD, przy \emph{ostrych stałych} i \emph{czasie bliskim liniowemu}.

\textbf{Przykład.} Weźmy macierz obrazów pikselowych \(A\in\RR^{100\times 100}\). Zamiast obliczać pełne SVD, wybieramy \(k=10\) kolumn (np. odpowiadających wybranym pikselom–kolumnom). CUR daje aproksymację, w której \(A\approx CUR\) z zachowaniem interpretacji „kolumn jako cech rzeczywistych”.

\textbf{Dlaczego to ważne?} Most między teorią a praktyką: interpretowalne składowe (kolumny) zamiast wektorów własnych; bezpośrednie zastosowania w CUR.

\textbf{Lektura startowa.} Klasyczne prace o CSSP (Deshpande–Vempala; Boutsidis–Mahoney–Drineas), nowsze algorytmy losowe i deterministyczne.
\end{openproblem}

\section{Struktury specjalne i stożki macierzowe}

\begin{openproblem}[Hipoteza o macierzach Hadamarda]\label{prob:Hadamard}
\textbf{Sformułowanie.} Czy dla każdego \(n\equiv 0\ (\mathrm{mod}\ 4)\) istnieje macierz Hadamarda rzędu \(n\)? Problem otwarty od ponad wieku.

\textbf{Przykład.} Dla \(n=4\) macierz Hadamarda to np.
\[
H_4 = \begin{bmatrix}
  1 &  1 &  1 &  1 \\
  1 & -1 &  1 & -1 \\
  1 &  1 & -1 & -1 \\
  1 & -1 & -1 &  1
\end{bmatrix}.
\]
Dla \(n=8\) znane są konstrukcje rekurencyjne (iloczyn Kroneckera macierzy \(H_2\) z \(H_4\)).

\textbf{Dlaczego to ważne?} Projektowanie eksperymentów, kody korekcyjne, teoria grafów i konstrukcje kombinatoryczne. Obecnie znane są konstrukcje dla wielu rzędów, ale pełna odpowiedź pozostaje nieznana.

\textbf{Lektura startowa.}
\begin{itemize}
  \item K. J. Horadam, \emph{Hadamard Matrices and Their Applications}, Princeton University Press (2007).
  \item M. Hall, \emph{Combinatorial Theory}, rozdziały o macierzach Hadamarda.
  \item Survey: J. Seberry, M. Yamada, \emph{Hadamard matrices, sequences, and block designs} (w: Contemporary Design Theory, 1992).
\end{itemize}

\textbf{Status.} Udowodniono istnienie dla wielu klas rzędów (np. potęgi 2, iloczyny pewnych rodzin liczb), ale ogólny przypadek pozostaje otwarty.
\end{openproblem}

% --- reszta sekcji ---

\section*{Mini–projekty dla studentów}
\begin{enumerate}[label=\textbf{P\arabic*}., leftmargin=8mm]
  \item \textbf{Eksperymenty z CSSP/CUR.} Porównaj losowe, dźwigniowe i deterministyczne wybory kolumn na danych rzeczywistych; zmierz błąd względem SVD i czas działania.
  \item \textbf{Osobliwość \(\pm1\).} Zaimplementuj testy osobliwości dla losowych macierzy Bernoulliego do rozmiarów \(n\approx 200\); spróbuj wykrywać typowe przyczyny osobliwości.
  \item \textbf{Sztywność w praktyce.} Dla różnych jawnych macierzy (Toeplitz, Vandermonde) szukaj heurystycznie niewielkich zmian obniżających rangę; raportuj zależności rozmiarowe.
  \item \textbf{Hadamardy i projektowanie.} Zbierz znane konstrukcje; wygeneruj małe przykłady i przetestuj własności ortogonalności numerycznie.
  \item \textbf{NIEP w małych wymiarach.} Spróbuj (symbolicznie/numerycznie) budować macierze nieujemne o zadanym prostym widmie; sformułuj hipotezy.
\end{enumerate}

\end{document}
