\documentclass[12pt,a4paper]{article}
\usepackage[utf8]{inputenc}
\usepackage[T1]{fontenc}
\usepackage{lmodern}
\usepackage{amsmath,amssymb}

\begin{document}

\begin{center}
\Large \textbf{Podsumowanie zajęć}
\end{center}

\bigskip

Zaczęliśmy od pozornie prostego zadania: „znajdź rozwiązanie układu równań”.
Dodanie trzeciego równania sprawiło, że układ stał się sprzeczny i nie miał rozwiązania w klasycznym sensie.

To otwiera ważne pytania:
\begin{itemize}
    \item Co znaczy „rozwiązać” problem, gdy dokładne spełnienie wszystkich równań jest niemożliwe?
    \item Jak rozumieć „najlepsze przybliżenie” — geometrycznie i obliczeniowo?
    \item Jakie narzędzia algebry liniowej pozwalają sformułować i znaleźć takie rozwiązania?
\end{itemize}

Dzisiejsze rozmowy z AI miały pokazać, że kluczowa jest \emph{umiejętność zadawania pytań}.
Dociekliwe pytania prowadzą do nowych pojęć i metod, a więc do głębszego rozumienia.

\bigskip
\begin{center}
\textit{„Matematyka zaczyna się tam, gdzie proste metody przestają działać.”}
\end{center}

\end{document}
