\documentclass[10pt,a4paper]{article}

\usepackage[utf8]{inputenc}
\usepackage[T1]{fontenc}
\usepackage{lmodern}
\usepackage{amsmath, amssymb, amsthm}
\usepackage{geometry}
\geometry{margin=2.5cm}

\title{Wprowadzenie do rozkładów macierzowych}
\author{Notatki z wykładu}
\date{}

\begin{document}
\maketitle

\section*{Cel wykładu}
Celem kursu jest zrozumienie metod rozkładu macierzy, które pozwalają 
odkrywać ukrytą strukturę danych. 
Będziemy rozważać zarówno teorię, jak i zastosowania praktyczne, 
takie jak kompresja obrazów, analiza danych czy rekomendacje w systemach informatycznych.

\section*{Dlaczego rozkłady macierzowe?}
Każda macierz może być traktowana jako mapa powiązań:
\begin{itemize}
    \item W statystyce macierz opisuje relacje między obiektami a ich cechami.
    \item W informatyce obraz cyfrowy to macierz pikseli.
    \item W analizie języka naturalnego duże macierze opisują zależności 
    między słowami a dokumentami.
\end{itemize}
Rozkłady macierzowe umożliwiają odkrycie ukrytej struktury, 
redukcję wymiaru i interpretację najważniejszych kierunków w danych.

\section*{Przykłady motywujące}
\begin{enumerate}
    \item \textbf{Tabela wyników studentów:} 
    wiersze odpowiadają studentom, kolumny przedmiotom. 
    Rozkład macierzy ujawnia podobieństwa między studentami i przedmiotami.
    
    \item \textbf{Obraz cyfrowy:} 
    macierz pikseli można przybliżyć prostszą strukturą, 
    co prowadzi do kompresji danych przy zachowaniu jakości obrazu.
    
    \item \textbf{Wyszukiwarki internetowe:} 
    macierze opisujące związki między zapytaniami i dokumentami 
    można rozłożyć, aby znaleźć najważniejsze informacje i szybciej odpowiadać na pytania użytkowników.
\end{enumerate}

\section*{Rozkłady, które poznamy}
\begin{itemize}
    \item \textbf{SVD} (Singular Value Decomposition) -- podstawowy rozkład, 
    który przedstawia macierz w postaci trzech czynników, 
    pozwalając na redukcję wymiaru i analizę głównych kierunków.
    
    \item \textbf{CUR decomposition} -- rozkład, w którym przybliżenie macierzy 
    opiera się na rzeczywistych kolumnach i wierszach, 
    co daje przewagę interpretacyjną.
\end{itemize}

\section*{Plan kursu}
Kurs będzie składał się z części teoretycznej i praktycznej:
\begin{enumerate}
    \item Powtórka z algebry liniowej: normy, iloczyny skalarne, 
    macierze ortogonalne.
    \item Definicja i własności rozkładu SVD.
    \item Zastosowania SVD: kompresja, PCA, redukcja wymiaru.
    \item CUR decomposition i porównanie z SVD.
    \item Zadania i projekty z rzeczywistymi danymi.
\end{enumerate}

\end{document}
