\documentclass[12pt]{article}

% --- Preambuła ---
\usepackage[utf8]{inputenc}
\usepackage[T1]{fontenc}
%\usepackage[polish]{babel}
\usepackage{lmodern}
\usepackage{geometry}
\geometry{margin=1in}
\usepackage{amsmath}
\let\lll\undefined % zapobiega konfliktowi z polskim pakietem
\usepackage{amssymb}
\usepackage{mathtools}
\usepackage{bm}
\usepackage{enumitem}
\usepackage{xcolor}
\usepackage{hyperref}
\hypersetup{colorlinks=true, linkcolor=blue, urlcolor=blue, citecolor=teal}
\usepackage{amsthm}
\usepackage{physics}
\usepackage{booktabs}
\usepackage{listings}
\usepackage{caption}

% --- Ustawienia listings (R) ---
\lstdefinestyle{Rstyle}{
  language=R,
  basicstyle=\ttfamily\small,
  keywordstyle=\color{blue}\bfseries,
  commentstyle=\color{gray!70}\itshape,
  stringstyle=\color{purple!70!black},
  numbers=left, numberstyle=\tiny, numbersep=8pt,
  showstringspaces=false,
  frame=single, rulecolor=\color{gray!50},
  breaklines=true, tabsize=2, columns=fullflexible,
  literate={ą}{{\k{a}}}1 {ć}{{\'c}}1 {ę}{{\k{e}}}1 {ł}{{\l{}}}1
           {ń}{{\'n}}1 {ó}{{\'o}}1 {ś}{{\'s}}1 {ż}{{\.z}}1 {ź}{{\'z}}1
           {Ą}{{\k{A}}}1 {Ć}{{\'C}}1 {Ę}{{\k{E}}}1 {Ł}{{\L{}}}1
           {Ń}{{\'N}}1 {Ó}{{\'O}}1 {Ś}{{\'S}}1 {Ż}{{\.Z}}1 {Ź}{{\'Z}}1,
}
\lstset{style=Rstyle}

% --- Środowiska twierdzeń ---
\newtheorem{defi}{Definicja}
\newtheorem{prop}{Własność}
\newtheorem{ex}{Przykład}
\newtheorem{exer}{Zadanie}

% --- Skróty ---
\newcommand{\R}{\mathbb{R}}
\newcommand{\Sp}{\mathrm{Sp}}
\newcommand{\Id}{\mathrm{Id}}

\title{Dlaczego macierze symplektyczne są ważne?\\ \large Krótka lekcja z zadaniami i ćwiczeniem w R}
\author{Kurs: Rozkłady macierzowe i zastosowania}
\date{}

\begin{document}
\maketitle

\section*{Cel lekcji}
Zrozumieć, czym jest forma symplektyczna, macierz symplektyczna i dlaczego transformata Fouriera może być postrzegana jako rotacja w przestrzeni fazowej. Zobaczyć na eksperymencie numerycznym (R), że \emph{symplectic Euler} zachowuje energię znacznie lepiej niż zwykły Euler w układach Hamiltonowskich (oscylator harmoniczny).

\section{Geometria symplektyczna w skrócie}
\begin{defi}[Przestrzeń symplektyczna]
Liniowa przestrzeń symplektyczna to para $(V,\omega)$, gdzie $V$ jest przestrzenią wektorową nad $\R$ wymiaru $2n$, zaś $\omega:V\times V\to\R$ jest antysymetryczna, czyli dla każdego $v,w\in V$, $\omega(v,u)=-\omega(u,v)$ i niezdegenerowana, czyli jeśli dla każdego $v\in V$, $\omega(v,u)=0$ to $u=0$.. W bazie kanonicznej
\[
\omega(u,v) = u^\top J v,\qquad 
J=\begin{bmatrix}0&I_n\\ -I_n&0\end{bmatrix}.
\]
\end{defi}
 Zwykle podaje się w definicji następujące warunki. 
\begin{defi} Definicja
 $\omega:V\times V\to\R$ jest dwuliniowa, alternująca czyli dla każdego $v\in V$, $\omega(v,v)=0$ oraz  niezdegenerowana
 \end{defi}
Pokazać, że antysymetryczność jest równoważna z własnością alternacji. Uwaga. Korzystamy tutaj, że pracujemy nad ciałem $R$. 
 
\begin{defi}[Rozmaitość symplektyczna]
Rozmaitość symplektyczna to para $(M,\omega)$, gdzie $M$ jest gładką rozmaitością wymiaru $2n$, a $\omega$ jest gładką, niezdegenerowaną i zamkniętą ($d\omega=0$) 2-formą. Na przestrzeni fazowej $T^*Q$ mamy kanoniczną 1-formę Liouville’a $\theta=\sum_i p_i\,dq_i$ oraz kanoniczną formę symplektyczną $\omega=d\theta=\sum_i dq_i\wedge dp_i$.
\end{defi}

\section{Macierze symplektyczne}
\begin{defi}
Macierz $A\in\R^{2n\times 2n}$ jest \emph{symplektyczna}, jeśli
\[
A^\top J A = J,\qquad 
J=\begin{bmatrix}0&I_n\\ -I_n&0\end{bmatrix}.
\]
Zbiór takich macierzy tworzy grupę $\Sp(2n,\R)$.
\end{defi}

\begin{prop}[Własności]
Jeśli $A\in\Sp(2n,\R)$, to:
\begin{enumerate}[label=(\alph*)]
    \item $A$ jest odwracalna, $A^{-1}=J^{-1}A^\top J = -J A^\top J$,
    \item $\det A=1$ (zachowanie objętości Liouvilla),
    \item widmo spełnia: jeśli $\lambda$ jest wartością własną, to również $\overline{\lambda}$ i $1/\lambda$.
\end{enumerate}
\end{prop}

\begin{ex}[Rotacja fazowa]
W przypadku $n=1$ macierz
\[
R=\begin{bmatrix}0&1\\ -1&0\end{bmatrix}
\]
spełnia $R^\top J R=J$ oraz działa na współrzędne $(q,p)$ jak $(q,p)\mapsto (p,-q)$, co jest rotacją o $90^\circ$ w płaszczyźnie fazowej.
\end{ex}

\section{Wiązka kostyczna i przekształcenia kanoniczne}
Niech $(\R^{2n},\omega)$ będzie standardową przestrzenią symplektyczną z
\[
\omega=\sum_{i=1}^n dq_i\wedge dp_i.
\]
Przekształcenie $\Phi:\R^{2n}\to\R^{2n}$ nazywamy \emph{kanonicznym} (symplektycznym), jeśli $\Phi^*\omega=\omega$.

Dla
\[
\Phi(q,p)=(p,-q)
\]
mamy $dq'_i=dp_i$ oraz $dp'_i=-dq_i$, zatem
\[
\Phi^*\omega = \sum_{i=1}^n dq'_i\wedge dp'_i
= \sum_{i=1}^n dp_i\wedge(-dq_i)
= \sum_{i=1}^n dq_i\wedge dp_i
= \omega,
\]
czyli $\Phi$ jest przekształceniem kanonicznym.

\section{Zadania rachunkowe}
\begin{exer}[Sprawdzenie symplektyczności]
Dla $n=1$ i
\[
A_\alpha=\begin{bmatrix}\cos\alpha&\sin\alpha\\ -\sin\alpha&\cos\alpha\end{bmatrix}
\]
pokaż, że $A_\alpha^\top J A_\alpha = J$ dla każdego $\alpha\in\R$. Wniosek?
\end{exer}

\begin{exer}[Warunek blokowy]
Niech 
\[
A=\begin{bmatrix} a& b\\ c& d\end{bmatrix},\quad a,b,c,d\in\R^{n\times n}.
\]
Wyprowadź równania równoważne $A^\top J A=J$, tzn.
\[
a^\top c=c^\top a,\quad b^\top d=d^\top b,\quad a^\top d-c^\top b=I_n.
\]
\end{exer}

\begin{exer}[Objętość Liouvilla]
Pokaż, że $A^\top J A=J$ implikuje $\det A=1$. (Wsk.: weź wyznacznik po obu stronach.)
\end{exer}

\begin{exer}[„Dlaczego to ma znaczenie?”]
Rozważ układ Hamiltona $H(q,p)=\tfrac12(p^2+\omega^2 q^2)$. Napisz równania Hamiltona i pokaż, że przepływ jest rotacją w płaszczyźnie $(q,p)$ z prędkością kątową $\omega$.
\end{exer}

\section{Ćwiczenie komputerowe w R: Euler vs. \emph{Symplectic Euler}}
\subsection*{Cel}
Porównać zachowanie energii numerycznej dla oscylatora $H(q,p)=\tfrac12(p^2+q^2)$:
\[
\dot q = \frac{\partial H}{\partial p}=p,\qquad 
\dot p = -\frac{\partial H}{\partial q}=-q.
\]
Zobaczymy, że metoda Eulera „ucieka” z energią, a \emph{symplectic Euler} zachowuje energię w długim czasie znacznie lepiej.

\subsection*{Kod R (uruchom w R/RStudio)}
\begin{lstlisting}
# --- Parametry ---
Tmax <- 200
h    <- 0.05
N    <- ceiling(Tmax/h)

# --- Warunki początkowe ---
q0 <- 1
p0 <- 0

# --- Pomocnicze: energia ---
H <- function(q, p) 0.5*(q^2 + p^2)

# --- 1) Zwykły Euler jawny ---
q_e <- numeric(N+1); p_e <- numeric(N+1); E_e <- numeric(N+1)
q_e[1] <- q0; p_e[1] <- p0; E_e[1] <- H(q_e[1], p_e[1])

for (k in 1:N) {
  q_e[k+1] <- q_e[k] + h * p_e[k]
  p_e[k+1] <- p_e[k] - h * q_e[k]
  E_e[k+1] <- H(q_e[k+1], p_e[k+1])
}

# --- 2) Symplectic Euler (wersja "p-then-q") ---
q_s <- numeric(N+1); p_s <- numeric(N+1); E_s <- numeric(N+1)
q_s[1] <- q0; p_s[1] <- p0; E_s[1] <- H(q_s[1], p_s[1])

for (k in 1:N) {
  # najpierw aktualizuj p używając q z chwili k
  p_s[k+1] <- p_s[k] - h * q_s[k]
  # potem q używając p z chwili k+1
  q_s[k+1] <- q_s[k] + h * p_s[k+1]
  E_s[k+1] <- H(q_s[k+1], p_s[k+1])
}

# --- Wykresy (energia w czasie) ---
op <- par(mfrow=c(1,2))
plot(seq(0, N)*h, E_e, type="l", xlab="t", ylab="Energia", main="Euler jawny: energia")
plot(seq(0, N)*h, E_s, type="l", xlab="t", ylab="Energia", main="Symplectic Euler: energia")
par(op)

# --- Tor w przestrzeni fazowej ---
op <- par(mfrow=c(1,2), asp=1)
plot(q_e, p_e, type="l", xlab="q", ylab="p", main="Euler jawny: tor (q,p)")
plot(q_s, p_s, type="l", xlab="q", ylab="p", main="Symplectic Euler: tor (q,p)")
par(op)

# --- Bonus: szybki test "symplektyczności" macierzy 2x2 ---
is_symplectic_2x2 <- function(A) {
  J <- matrix(c(0,1,-1,0), 2, 2)
  all(abs(t(A) %*% J %*% A - J) < 1e-10)
}

R90 <- matrix(c(0,1,-1,0), 2, 2)
is_symplectic_2x2(R90)  # powinno być TRUE

# Przykład rotacji o kąt alpha:
alpha <- 0.37
A <- matrix(c(cos(alpha), sin(alpha), -sin(alpha), cos(alpha)), 2, 2)
is_symplectic_2x2(A)    # TRUE
\end{lstlisting}

\paragraph{Komentarz: Euler zwykły a Euler symplektyczny.}

Rozważany układ Hamiltona
\[
H(q,p)=\tfrac{1}{2}(q^2+p^2), \qquad 
\dot q = p, \quad \dot p = -q
\]
ma rozwiązania dokładne poruszające się po okręgu w płaszczyźnie $(q,p)$, a energia
\[
E(t) = \tfrac{1}{2}\big(q(t)^2+p(t)^2\big)
\]
jest zachowana.

\begin{itemize}
\item \textbf{Euler jawny} (klasyczny schemat Eulera):
\[
q_{k+1} = q_k + h p_k, \qquad 
p_{k+1} = p_k - h q_k,
\]
nie respektuje struktury Hamiltonowskiej. W krótkim czasie energia może chwilowo maleć lub rosnąć w zależności od warunków początkowych, 
ale w dłuższym okresie zawsze \emph{rośnie wykładniczo}, a tor numeryczny spiralnie oddala się od okręgu.

\item \textbf{Euler symplektyczny} (tzw. \emph{symplectic Euler}):

Schemat \textbf{Eulera symplektycznego} (wariant $p$-then-$q$) ma postać
\[
\begin{cases}
p_{k+1} = p_k - h q_k, \\
q_{k+1} = q_k + h p_{k+1}.
\end{cases}
\]

Możemy to zapisać w postaci macierzowej:
\[
\begin{bmatrix} q_{k+1} \\ p_{k+1} \end{bmatrix}
=
M_{\text{symp}}
\begin{bmatrix} q_k \\ p_k \end{bmatrix},
\qquad 
M_{\text{symp}} = 
\begin{bmatrix}
1 & h \\ -h & 1-h^2
\end{bmatrix}.
\]
\end{itemize}

\begin{exer}[Sprawdzenie symplektyczności]
Przypomnijmy: macierz $M$ jest symplektyczna, jeśli
\[
M^\top J M = J, 
\qquad 
J=\begin{bmatrix}0 & 1 \\ -1 & 0\end{bmatrix}.
\]
Zweryfikuj rachunkowo, że $M_{\text{symp}}$ spełnia ten warunek, a macierz kroku zwykłego Eulera
\[
M_{\text{Euler}} = \begin{bmatrix}1 & h \\ -h & 1\end{bmatrix}
\]
go nie spełnia.
\end{exer}

\noindent
Dlatego metody symplektyczne są preferowane w długoterminowych symulacjach układów Hamiltonowskich (np. w astronomii, dynamice molekularnej): zamiast dokładności „punkt po punkcie” gwarantują poprawne odwzorowanie \emph{geometrii} ruchu i zachowanie własności jakościowych (stabilność, energia w przybliżeniu stała).

\section*{Kącik angielski (terminy kluczowe)}
\begin{itemize}[label=$\triangleright$]
\item symplectic form — forma symplektyczna
\item canonical transformation — przekształcenie kanoniczne
\item phase space — przestrzeń fazowa
\item Liouville 1-form / Liouville form — forma Liouville’a
\item symplectic integrator — całkownik symplektyczny
\item Fourier transform as phase-space rotation — transformata Fouriera jako rotacja w przestrzeni fazowej
\end{itemize}

\section*{Literatura (na start)}
\begin{enumerate}
\item V.\,I. Arnold, \emph{Mathematical Methods of Classical Mechanics}.
\item R.\,Abraham, J.\,E. Marsden, \emph{Foundations of Mechanics}.
\item H. Goldstein, C.\,P. Poole, J.\,L. Safko, \emph{Classical Mechanics} (rozdz. Hamiltonowskie).
\item E. Hairer, C. Lubich, G. Wanner, \emph{Geometric Numerical Integration}.
\end{enumerate}

\section*{Mini-quiz (1–2 min)}
\begin{enumerate}[label=\arabic*.]
\item Podaj definicję macierzy symplektycznej w postaci $A^\top J A=J$.
\item Dlaczego $\det A=1$ dla $A\in\Sp(2n,\R)$?
\item Jak działa na $(q,p)$ macierz $R=\begin{bmatrix}0&1\\ -1&0\end{bmatrix}$?
\end{enumerate}

\end{document}
