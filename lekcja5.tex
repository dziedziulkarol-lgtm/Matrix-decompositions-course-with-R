\documentclass[12pt]{article}
\usepackage[utf8]{inputenc}
\usepackage[T1]{fontenc}
\usepackage{lmodern}
\usepackage{geometry}
\geometry{margin=1in}
\usepackage{amsmath,amssymb}
\usepackage{amsthm}
\usepackage{enumitem}
\usepackage{xcolor}
\usepackage{hyperref}
\usepackage{graphicx}

% środowisko do zadań
\newtheorem{zadanie}{Zadanie}[section]

\title{Lekcja 5: Problemy otwarte i motywacja do dalszych rozkładów macierzowych}
\author{}
\date{}

\begin{document}
\maketitle

\begin{quote}
\emph{“Not every system has a solution, not every matrix is invertible. 
But every matrix hides structure — and we need tools to uncover it.”}
\end{quote}

\section*{Dlaczego problemy otwarte?}
W algebrze liniowej klasyczne twierdzenia często zakładają warunki idealne 
(np. macierz kwadratowa, niezerowy wyznacznik). 
W praktyce:
\begin{itemize}
  \item układy równań bywają sprzeczne,
  \item macierze są prostokątne i zdegenerowane,
  \item dane zawierają szum.
\end{itemize}
Dlatego przechodzimy od teorii klasycznej do narzędzi ogólnych, 
jak rozkład SVD (Singular Value Decomposition).

\section*{Definicja SVD}
Niech \(A\in\mathbb{R}^{m\times n}\). Istnieją macierze ortogonalne 
\(U\in\mathbb{R}^{m\times m}\), \(V\in\mathbb{R}^{n\times n}\)
oraz diagonalna \(\Sigma\in\mathbb{R}^{m\times n}\) z nieujemnymi elementami
\(\sigma_1\ge \sigma_2\ge \cdots \ge 0\) takie, że
\[
A=U\,\Sigma\,V^\top.
\]

\section{  SVD i macierz Hadamarda}
\begin{itemize}
  \item Pytanie otwarte: czy dla każdego rzędu $n$ istnieje macierz Hadamarda?
  \item Znane tylko dla $n=1,2$ i wielokrotności $4$ do pewnego rozmiaru.
  \item macierz Hadamarda: macierz $\pm1$ o ortogonalnych wierszach.
\end{itemize}

\textbf{Uwaga:} jeżeli \(H_n\) istnieje, to \(\tfrac{1}{\sqrt{n}}H_n\) jest macierzą ortogonalną.  
Dla \(H_4\) można sprawdzić, że
\[
H_4H_4^\top = 4I.
\]

\begin{zadanie}
Zweryfikuj powyższe równanie.
Jaki jest najprostszy rozkład SVD dla macierzy $H_4$?
Czy potrafisz napisać wzory dla dowolnej macierzy \(H_n\)?
\end{zadanie}

To przykład, gdzie SVD jest trywialne — wszystkie wartości osobliwe są równe 
$\sqrt{n}$. Pokazuje to, że nie zawsze SVD jest narzędziem odkrywczym; czasem potwierdza tylko znaną symetrię.

\section{ Zastosowanie SVD} 
\[
A = \begin{bmatrix}
  3 &  1 &  1 &  1 \\
  1 & 0 &  1 & 0 \\
  1 &  1 & 0 & 0 \\
  1 & 0 & 0 &  1
\end{bmatrix}.
\]
\begin{zadanie}
Wyznacz pseudoodwrotną macierz \(A^\dagger\).  
Sprawdź wynik na dwa sposoby:
\begin{enumerate}
  \item bezpośrednio z definicji pseudoodwrotności,
  \item przez rozkład SVD: $A = U\Sigma V^\top$, a następnie
$A^\dagger = V\Sigma^\dagger U^\top$, 
gdzie $\Sigma^\dagger$ powstaje przez odwrócenie dodatnich 
(niezerowych) wartości osobliwych i ma rozmiar $n\times m$. Zaobserwuj to w przykładzie.
\end{enumerate}
\end{zadanie}

\medskip

\section{ Sprzeczny układ równań}
Rozważmy układ
\[
x+y=1,\qquad x+y=2,\qquad x-y=0.
\]
W zapisie macierzowym:
\[
A=\begin{bmatrix}1 & 1 \\ 1 & 1 \\ 1 & -1\end{bmatrix}, \quad
b=\begin{bmatrix}1 \\ 2 \\ 0\end{bmatrix}.
\]

\begin{itemize}
  \item Układ nie ma rozwiązania dokładnego.
  \item Jak znaleźć „najlepsze przybliżenie”?
\end{itemize}

\textbf{Odpowiedź:} metoda najmniejszych kwadratów, pseudoodwrotność, rzut na podprzestrzeń.

\begin{zadanie}
Wyznacz macierz pseudoodwrotną \(A^\dagger\) metodą SVD oraz rozwiązanie 
\(x^\dagger = A^\dagger b\). 
\end{zadanie}

Pseudoodwrotność daje punkt „najbliższy” wszystkim trzem równaniom w sensie LSQ,
błąd średniokwadratowy.

\begin{zadanie}
Wspólnie z AI napisz działający program w~R, który rozwiązuje problem optymalizacji
\[
x^\dagger=\operatorname*{argmin}_{x\in\mathbb{R}^2}\|Ax-b\|_2.
\] 
\end{zadanie}

\medskip

\section{ Kompresja obrazu}
Obraz cyfrowy można zapisać jako macierz pikseli (np. w skali szarości).
\begin{itemize}
  \item Jak zapisać obraz przybliżony mniejszą liczbą danych?
  \item Czy da się znaleźć \emph{najlepszą} macierz o zadanej randze $k$?
\end{itemize}
\textit{Wskazówka:} Aproksymacja rangi $k$ to 
$A_k = \sum_{i=1}^{k} \sigma_i u_i v_i^\top$, 
a tw. Eckarta–Younga mówi, że $A_k$ minimalizuje $\|A - X\|_F$ wśród $\operatorname{rank}(X)\le k$.


\begin{zadanie}
Zredukuj rangę macierzy \(A\) korzystając z~SVD.  
Porównaj ilość danych potrzebnych do przechowania macierzy pełnej i macierzy o zredukowanej randze.
\end{zadanie}

\begin{zadanie}
(Doświadczalne) Wczytaj w R lub Pythonie prosty obraz (np. czarno-biały kwadrat \(8\times 8\)).  
Zastosuj SVD i zapisz obraz przybliżony rangą \(k=1,2,3\).  
Jak zmienia się jakość wizualna?
\end{zadanie}

\medskip

\section{ Non-negative matrix factorization (NMF)}
Załóżmy, że mamy dane tylko z wartościami nieujemnymi (np. liczba słów w dokumentach, intensywności pikseli), np.
\[
A = \begin{bmatrix}
  3 &  1 &  1 &  1 \\
  1 & 0 &  1 & 0 \\
  1 &  1 & 0 & 0 \\
  1 & 0 & 0 &  1
\end{bmatrix}.
\]
\begin{itemize}
  \item Czy można przybliżyć taką macierz iloczynem dwóch macierzy nieujemnych?
  \item Taki rozkład istnieje, ale nie zawsze jest jednoznaczny.
\end{itemize}

\textbf{Zastosowania:} analiza tekstu, rozpoznawanie obrazów, biologia obliczeniowa.  

\begin{zadanie}
Wyznacz (numerycznie) rozkład macierzy \(A\) na macierze nieujemne.  
Zastanów się, dlaczego wymóg nieujemności ma sens w analizie danych (np. słowa, piksele). 
\end{zadanie}
Pytanie do grupy: „Dlaczego SVD może dać wartości ujemne, a NMF nie?” —  proszę o komentarz grupy, dyskusję.

\textit{Wskazówka:} Sformułuj problem jako 
$\min_{W,H\ge 0}\ \|A-WH\|_F^2$ przy zadanej randze wewnętrznej $r$.
\medskip

\section*{Podsumowanie lekcji 5}
Problemy otwarte i praktyczne motywują nas do rozkładów macierzowych:
\begin{itemize}
  \item SVD — zawsze działa, daje pełny obraz struktury,
  \item inne rozkłady (NMF, CUR) — użyteczne w analizie danych,
  \item inspiracja do dalszych badań i zastosowań.
\end{itemize}
Widzimy więc trzy poziomy zastosowania:
\begin{itemize}
\item  struktury matematyczne (Hadamard),
\item  rozwiązywanie równań (sprzeczne układy),
\item analiza danych (obrazy, tekst).
\end{itemize}
Wszystko łączy jedno narzędzie — rozkłady macierzowe.

\section*{Pytanie do grupy}
Jakie inne sytuacje (poza matematyką) znacie, gdzie nie ma „dokładnego rozwiązania”, 
ale szukamy najlepszego przybliżenia?

\end{document}
