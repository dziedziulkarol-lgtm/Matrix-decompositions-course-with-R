\documentclass[12pt]{article}

\usepackage[utf8]{inputenc}
\usepackage[T1]{fontenc}
\usepackage[polish]{babel}
\usepackage{lmodern}
\usepackage{geometry}
\geometry{margin=1in}
\let\lll\undefined
\usepackage{amsmath,amssymb,amsthm}
\usepackage{enumitem}
\usepackage{xcolor}
\usepackage{hyperref}


% --- Środowiska ---
\newtheorem{defi}{Definicja}
\newtheorem{theo}{Twierdzenie}
\newtheorem{prop}{Własność}
\newtheorem{ex}{Przykład}
\newtheorem{exer}{Zadanie}
\newcommand{\Sp}{\mathrm{Sp}}

\title{Transformata Fouriera, operatory $Q$ i $P$ oraz grupa symplektyczna}
\author{}
\date{}

\begin{document}
\maketitle

\section{Transformata Fouriera na $L^2$}
\begin{defi}
Dla $f\in S(\mathbb{R})$ (funkcje Schwartzowskie) definiujemy
\[
(\mathcal{F}f)(p) = \frac{1}{\sqrt{2\pi}}\int_{\mathbb{R}} f(q)\, e^{-i p q}\, dq.
\]
\end{defi}

\begin{theo}[Plancherel]
Transformata Fouriera $\mathcal{F}$ rozszerza się jednoznacznie do operatora unitarnego
\[
\mathcal{F}:L^2(\mathbb{R})\to L^2(\mathbb{R}).
\]
\end{theo}

\section{Dwa naturalne operatory}
\begin{defi}
Na $\mathcal{H}=L^2(\mathbb{R})$ definiujemy:
\begin{align*}
(Qf)(x) &= x f(x), & \mathrm{Dom}(Q)&=\{f: xf(x)\in L^2\},\\
(Pf)(x) &= -i f'(x), & \mathrm{Dom}(P)&=H^1(\mathbb{R}).
\end{align*}
\end{defi}

\begin{prop}[Relacja kanoniczna]
Na odpowiedniej gęstej dziedzinie zachodzi
\[
[Q,P]=QP-PQ = i I.
\]
\end{prop}

\begin{prop}[Przestrzeń wspólnej definicji]
Najwygodniej pracować na przestrzeni Schwartzowskiej $S(\mathbb{R})$, która jest gęsta w $L^2$ i stabilna względem działania $Q$ i~$P$.
Dla $f\in S(\mathbb{R})$ zachodzi dokładna równość
\[
(QP - PQ)f = i f.
\]
\end{prop}

\paragraph{Nawias Poissona (klasyka).} 
Na $\mathbb{R}^{2n}$ z współrzędnymi $(q,p)$ i formą symplektyczną 
\[
\omega = \sum_{i=1}^n dq_i\wedge dp_i
\]
definiujemy dla gładnych funkcji $f,g\in C^\infty(\mathbb{R}^{2n})$ nawias
\[
\{f,g\} = \sum_{i=1}^n 
\left( \frac{\partial f}{\partial q_i}\frac{\partial g}{\partial p_i}
     - \frac{\partial f}{\partial p_i}\frac{\partial g}{\partial q_i}\right).
\]
W mechanice kwantowej relacja $[Q,P]=iI$ jest dyskretnym odpowiednikiem klasycznego faktu $\{q,p\}=1$.
\section{Fourier a operatory $Q$ i $P$}
\begin{prop}
Zachodzi
\[
\mathcal{F} Q \mathcal{F}^{-1} = P, \qquad 
\mathcal{F} P \mathcal{F}^{-1} = -Q.
\]
\end{prop}

\begin{proof}[Idea dowodu]
Dla $f\in S(\mathbb{R})$ mamy
\[
(\mathcal{F}(Qf))(p) = \frac{1}{\sqrt{2\pi}}\int x f(x) e^{-ipx}\,dx
= i\frac{d}{dp}(\mathcal{F}f)(p),
\]
co daje pierwszą równość. Analogicznie dla $P$.
\end{proof}

\section{Interpretacja symplektyczna}
Rozważmy przestrzeń liniową $\mathbb{R}^2$ ze współrzędnymi $(q,p)$ i formą symplektyczną
\[
\omega((q,p),(q',p')) = q p' - p q'.
\]

\begin{defi}
Grupa symplektyczna $\Sp(2,\mathbb{R})$ to zbiór macierzy $A\in GL(2,\mathbb{R})$ takich, że
\[
A^\top J A = J, \qquad 
J=\begin{bmatrix}0&1\\ -1&0\end{bmatrix}.
\]
\end{defi}

\begin{ex}
Macierz
\[
R=\begin{bmatrix}0&1\\ -1&0\end{bmatrix}
\]
należy do $\Sp(2,\mathbb{R})$ i odpowiada rotacji $(q,p)\mapsto(p,-q)$.
\end{ex}

\section{Most między analizą a geometrią}
\begin{itemize}
\item Fourier na $L^2$ jest operatorem unitarnym.
\item Relacje $\mathcal{F}Q\mathcal{F}^{-1}=P$, $\mathcal{F}P\mathcal{F}^{-1}=-Q$ pokazują, że Fourier zamienia operator mnożenia i różniczkowania.
\item Algebraicznie to odpowiada rotacji $(q,p)\mapsto(p,-q)$ w grupie $\Sp(2,\mathbb{R})$.
\item Formalnie: transformata Fouriera jest realizacją tej rotacji w tzw. reprezentacji metaplektycznej.
\end{itemize}

\section{Zadania dla studentów}
\begin{exer}
Udowodnij, że dla $f(x)=e^{-x^2/2}$ zachodzi $\mathcal{F}f=f$. 
\end{exer}

\begin{exer}
Sprawdź bezpośrednio, że $\mathcal{F}Q\mathcal{F}^{-1}=P$. 
\end{exer}

\begin{exer}
Pokaż, że macierz rotacji $R$ należy do $\Sp(2,\mathbb{R})$, czyli $R^\top J R=J$. 
\end{exer}

\begin{exer}
Dla chętnych: sprawdź, że $\det A=1$ dla każdego $A\in \Sp(2,\mathbb{R})$. 
\end{exer}

\section*{Kącik angielski}
\begin{itemize}[label=$\triangleright$]
\item position operator — operator położenia ($Q$)
\item momentum operator — operator pędu ($P$)
\item canonical commutation relation — relacja kanoniczna komutacyjna
\item symplectic group — grupa symplektyczna
\item metaplectic representation — reprezentacja metaplektyczna
\end{itemize}

\section*{Literatura}
\begin{enumerate}
\item E. Stein, R. Shakarchi, \emph{Fourier Analysis: An Introduction}.
\item V. I. Arnold, \emph{Mathematical Methods of Classical Mechanics}.
\item M. Reed, B. Simon, \emph{Methods of Modern Mathematical Physics}, vol. I.
\end{enumerate}

\end{document}
